\documentclass[a4 paper]{article}
\usepackage[brazil]{babel}
\usepackage[utf8]{inputenc}
% Set target color model to RGB
\usepackage[inner=2.0cm,outer=2.0cm,top=2.5cm,bottom=2.5cm]{geometry}
\usepackage{setspace}
\usepackage[rgb]{xcolor}
\usepackage{verbatim}
\usepackage{subcaption}
\usepackage{amsgen,amsmath,amstext,amsbsy,amsopn,amssymb}
\usepackage{fancyhdr}
\usepackage[colorlinks=true, urlcolor=blue,  linkcolor=blue, citecolor=blue]{hyperref}
\usepackage[colorinlistoftodos]{todonotes}
\usepackage{rotating}
%\usetikzlibrary{through,backgrounds}
\hypersetup{%
pdfauthor={Ashudeep Singh},%
pdftitle={Homework},%
pdfkeywords={Tikz,latex,bootstrap,uncertaintes},%
pdfcreator={PDFLaTeX},%
pdfproducer={PDFLaTeX},%
}
%\usetikzlibrary{shadows}
% \usepackage[francais]{babel}
\usepackage{booktabs}
\usepackage{xcolor}
% Definindo novas cores
\definecolor{verde}{rgb}{0,0.5,0}
% Configurando layout para mostrar codigos C++
\usepackage{listings}
\lstset{
  language=C++,
  basicstyle=\ttfamily\small, 
  keywordstyle=\color{blue}, 
  stringstyle=\color{verde}, 
  commentstyle=\color{red}, 
  extendedchars=true, 
  showspaces=false, 
  showstringspaces=false, 
  numbers=left,
  numberstyle=\tiny,
  breaklines=true, 
  backgroundcolor=\color{green!10},
  breakautoindent=true, 
  captionpos=b,
  xleftmargin=0pt,
}

%%%%%% macros.tex
\newtheorem{thm}{Theorem}[section]
\newtheorem{prop}[thm]{Proposition}
\newtheorem{lem}[thm]{Lemma}
\newtheorem{cor}[thm]{Corollary}
\newtheorem{defn}[thm]{Definition}
\newtheorem{rem}[thm]{Remark}
\numberwithin{equation}{section}

\newcommand{\homework}[6]{
   \pagestyle{myheadings}
   \thispagestyle{plain}
   \newpage
   \setcounter{page}{1}
   \noindent
   \begin{center}
   \framebox{
      \vbox{\vspace{2mm}
    \hbox to 6.28in { {\bf Introdução à Análise de dados em FAE \hfill {\small (#2)}} }
       \vspace{6mm}
       \hbox to 6.28in { {\Large \hfill #1  \hfill} }
       \vspace{6mm}
       \hbox to 6.28in { {\it Professores: {\rm #3} \hfill Name: {\rm #5} } }
       %\hbox to 6.28in { {\it TA: #4  \hfill #6}}
      \vspace{2mm}}
   }
   \end{center}
   \markboth{#5 -- #1}{#5 -- #1}
   \vspace*{4mm}
}

\newcommand{\problem}[2]{~\\\fbox{\textbf{#1}}\hfill \newline\newline}
\newcommand{\subproblem}[1]{~\newline\textbf{(#1)}}
\newcommand{\D}{\mathcal{D}}
\newcommand{\Hy}{\mathcal{H}}
\newcommand{\VS}{\textrm{VS}}
\newcommand{\solution}{~\newline\textbf{\textit{(Solução)}} }

\newcommand{\bbF}{\mathbb{F}}
\newcommand{\bbX}{\mathbb{X}}
\newcommand{\bI}{\mathbf{I}}
\newcommand{\bX}{\mathbf{X}}
\newcommand{\bY}{\mathbf{Y}}
\newcommand{\bepsilon}{\boldsymbol{\epsilon}}
\newcommand{\balpha}{\boldsymbol{\alpha}}
\newcommand{\bbeta}{\boldsymbol{\beta}}
\newcommand{\0}{\mathbf{0}}
%%%% macros.tex

\begin{document}
\homework{Exercícios - Aula ROOT}{15/10/2024}{Dilson, Eliza \& Maurício}{}{Diogo Gomes Santos Caffonso de Morais}

\problem{EXERCÍCIO 1 }

O código para o exercício foi:

\begin{lstlisting}
#include <iostream>
#include <cmath>
#include "TCanvas.h"
#include "TF1.h"
#include "TGraph.h"
#include "TAxis.h"
#include "TMath.h"
#include "Math/Integrator.h"

double myFunc(double *x, double *par) {
    double p0 = par[0];
    double p1 = par[1];
    return p0 * TMath::Sin(p1 * x[0]) / x[0];
}

void exercicio1() {
        double p0 = 1.0;
        double p1 = 2.0;
        double x = 1.0;

        TCanvas *c1 = new TCanvas("c1", "Function Plot", 800, 600);
        TF1 *f1 = new TF1("f1", myFunc, 0.01, 10, 2);
        f1->SetParameters(p0,p1);
        f1->SetLineColor(kBlue);
        f1->Draw();
        c1->SaveAs("function_plot.png");

        double func_value = f1->Eval(x);
        std::cout << "Function value at x = 1: " << func_value << std::endl;

        double dx = 1e-6;
        double derivada = (f1->Eval(x + dx) - f1->Eval(x - dx)) / (2 * dx);
        std::cout << "Derivative at x = 1: " << derivada << std::endl;

        double integral = f1->Integral(0, 3);
        std::cout << "Integral from 0 to 3: " << integral << std::endl;
}

\end{lstlisting}

Executado o código acima, tive o seguinte output:

\includegraphics[width=\linewidth]{ex1.png}
\begin{center}
    Figura 1: Gráfico da função $f_{1}(x)$.
\end{center}

Seguindo o comando do exercício:\large $$f_{1}(1)=0,909297$$ $$f_{1}^{'}(1)=-1,74159$$ $$\int_{0}^{3}f_{1}(x)dx=1,42469$$\normalsize

\newpage

\problem{EXERCÍCIO 2 }

O código para o exercício foi:

\begin{lstlisting}

#include <iostream>
#include <cmath>
#include "TCanvas.h"
#include "TF1.h"
#include "TGraph.h"
#include "TAxis.h"
#include "TMath.h"
#include "Math/Integrator.h"
#include "TGraphErrors.h"
#include "TStyle.h"

void exercicio2() {
	std::ifstream dataFile("graphdata.txt");
	std::ifstream errorFile("graphdata_error.txt");

	const int nPoints = 10;
	double x[nPoints], y[nPoints];
	double ex[nPoints], ey[nPoints];

	for (int i = 0; i < nPoints; ++i) {
		dataFile >> x[i] >> y[i];
	}

	for (int i = 0; i < nPoints; ++i) {
		errorFile >> x[i] >> y[i] >> ex[i] >> ey[i];
	}

	dataFile.close();
	errorFile.close();

	TCanvas *c1 = new TCanvas("c1", "Graph of Points", 800, 600);
	TGraph *graph1 = new TGraph(nPoints, x, y);
	graph1->SetMarkerStyle(21);
	graph1->SetMarkerColor(kBlack);
	graph1->SetTitle("Graph of Points;X;Y");
	graph1->Draw("AP");
	c1->SaveAs("graph_points.png");

	TCanvas *c2 = new TCanvas("c2", "Graph with Line", 800, 600);
	TGraph *graph2 = new TGraph(nPoints, x, y);
	graph2->SetMarkerStyle(21);
	graph2->SetMarkerColor(kBlack);
	graph2->SetLineColor(kBlue);
	graph2->SetLineWidth(2);
	graph2->SetTitle("Graph with Line;X;Y");
	graph2->Draw("APL");
	c2->SaveAs("graph_with_line.png");

	TCanvas *c3 = new TCanvas("c3", "Graph with Errors", 800, 600);
	TGraphErrors *graph3 = new TGraphErrors(nPoints, x, y, ex, ey);
	graph3->SetMarkerStyle(21);
	graph3->SetMarkerColor(kBlack);
	graph3->SetLineColor(kRed);
	graph3->SetLineWidth(2);
	graph3->SetTitle("Graph with Errors;X;Y");
	graph3->Draw("AP");
	c3->SaveAs("graph_with_errors.png");
}

\end{lstlisting}

Executado o código acima, tive o seguinte output:

\includegraphics[width=\linewidth]{ex21.png}
\begin{center}
    Figura 2: Gráfico de distribuição para os pontos.
\end{center}

\includegraphics[width=\linewidth]{ex23.png}
\begin{center}
    Figura 3: Gráfico de distribuição para os pontos, com linha que liga os pontos.
\end{center}

\includegraphics[width=\linewidth]{ex22.png}
\begin{center}
    Figura 4: Gráfico de distribuição para os pontos, com barra de erros.
\end{center}

\newpage

\problem{EXERCÍCIO 3 }

O código para o exercício foi:

\begin{lstlisting}
#include <iostream>
#include <cmath>
#include "TCanvas.h"
#include "TF1.h"
#include "TGraph.h"
#include "TAxis.h"
#include "TMath.h"
#include "Math/Integrator.h"
#include "TRandom.h"

void exercicio3() {

	TH1F *h1 = new TH1F("h1", "Gaussian Distribution", 50, 0, 10);

	TRandom *rand = new TRandom();
	for (int i = 0; i < 10000; i++) {
		h1->Fill(rand->Gaus(5, 2));  // Mean 5, Sigma 2
	}

	gStyle->SetOptStat("nemruoiks");
	gStyle->SetStatX(0.9);
	gStyle->SetStatY(0.9);
	gStyle->SetStatW(0.2);
	gStyle->SetStatH(0.3);

	TCanvas *c1 = new TCanvas("c1", "Gaussian Histogram", 800, 600);

	h1->Draw();

	c1->Update();

	TPaveStats *stats = (TPaveStats*)h1->GetListOfFunctions()->FindObject("stats");

	if (stats) {
		stats->SetName("mystats");

		double skewness = h1->GetSkewness();
		double kurtosis = h1->GetKurtosis();

		stats->AddText(Form("Skewness = %.3f", skewness));
		stats->AddText(Form("Kurtosis = %.3f", kurtosis));

		stats->SetX1NDC(0.7);
		stats->SetY1NDC(0.7);

		c1->Modified();
		c1->Update();
	}

	c1->SaveAs("gaussian_histogram.png");
}
\end{lstlisting}

Executado o código acima, tive o seguinte output:

\includegraphics[width=\linewidth]{ex3.png}
\begin{center}
    Figura 5: Histograma com os dados no statistic box.
\end{center}

\newpage

\problem{EXERCÍCIO 4 }

O código para o exercício foi:

\begin{lstlisting}
#include <iostream>
#include <cmath>
#include "TCanvas.h"
#include "TF1.h"
#include "TGraph.h"
#include "TAxis.h"
#include "TMath.h"
#include "Math/Integrator.h"
#include "TRandom.h"

void exercicio4() {

	TCanvas *c1 = new TCanvas("c1", "Total Momentum Distribution", 800, 600);
	TFile *file = TFile::Open("tree.root");
	TTree *tree = (TTree*)file->Get("tree1");
	TH1F *h1 = new TH1F("h1", "Total Momentum Distribution", 50, 0, 1000);
	TH1F *h2 = new TH1F("h2", "Cut Momentum Distribution", 100, 100, 200);
	float px, py, pz, e;
	Int_t noE = 1e+5;

	tree->SetBranchAddress("px", &px);
	tree->SetBranchAddress("py", &py);
	tree->SetBranchAddress("pz", &pz);
	tree->SetBranchAddress("ebeam", &e);

	for (Int_t i=0; i<noE; i++){
		tree->GetEntry(i);
		h1->Fill(e);
	}

	float e_mean = h1->GetMean();
	for (Int_t i=0; i<noE; i++){
		tree->GetEntry(i);
		if (e > e_mean + 0.2){
			float p = sqrt(px*px + py*py + pz*pz);
			h2->Fill(p);
		}	
	}

	h2->Draw();
	c1->SaveAs("total_momentum_distribution.png");

}
\end{lstlisting}

Executado o código acima, tive o seguinte output:

\includegraphics[width=\linewidth]{ex4.png}
\begin{center}
    Figura 6: Histograma de momento total para a janela de energia sugerida.
\end{center}


\end{document} 
