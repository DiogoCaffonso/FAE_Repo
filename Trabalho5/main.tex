\documentclass[a4 paper]{article}
\usepackage[brazil]{babel}
\usepackage[utf8]{inputenc}
% Set target color model to RGB
\usepackage[inner=2.0cm,outer=2.0cm,top=2.5cm,bottom=2.5cm]{geometry}
\usepackage{setspace}
\usepackage[rgb]{xcolor}
\usepackage{verbatim}
\usepackage{subcaption}
\usepackage{amsgen,amsmath,amstext,amsbsy,amsopn,amssymb}
\usepackage{fancyhdr}
\usepackage[colorlinks=true, urlcolor=blue,  linkcolor=blue, citecolor=blue]{hyperref}
\usepackage[colorinlistoftodos]{todonotes}
\usepackage{rotating}
\usepackage{hyperref}
%\usetikzlibrary{through,backgrounds}
\hypersetup{%
pdfauthor={Ashudeep Singh},%
pdftitle={Homework},%
pdfkeywords={Tikz,latex,bootstrap,uncertaintes},%
pdfcreator={PDFLaTeX},%
pdfproducer={PDFLaTeX},%
}
%\usetikzlibrary{shadows}
% \usepackage[francais]{babel}
\usepackage{booktabs}
\usepackage{xcolor}
% Definindo novas cores
\definecolor{verde}{rgb}{0,0.5,0}
% Configurando layout para mostrar codigos C++
\usepackage{listings}
\lstset{
  language=C++,
  basicstyle=\ttfamily\small, 
  keywordstyle=\color{blue}, 
  stringstyle=\color{verde}, 
  commentstyle=\color{red}, 
  extendedchars=true, 
  showspaces=false, 
  showstringspaces=false, 
  numbers=left,
  numberstyle=\tiny,
  breaklines=true, 
  backgroundcolor=\color{green!10},
  breakautoindent=true, 
  captionpos=b,
  xleftmargin=0pt,
}

%%%%%% macros.tex
\newtheorem{thm}{Theorem}[section]
\newtheorem{prop}[thm]{Proposition}
\newtheorem{lem}[thm]{Lemma}
\newtheorem{cor}[thm]{Corollary}
\newtheorem{defn}[thm]{Definition}
\newtheorem{rem}[thm]{Remark}
\numberwithin{equation}{section}

\newcommand{\homework}[6]{
   \pagestyle{myheadings}
   \thispagestyle{plain}
   \newpage
   \setcounter{page}{1}
   \noindent
   \begin{center}
   \framebox{
      \vbox{\vspace{2mm}
    \hbox to 6.28in { {\bf Introdução à Análise de dados em FAE \hfill {\small (#2)}} }
       \vspace{6mm}
       \hbox to 6.28in { {\Large \hfill #1  \hfill} }
       \vspace{6mm}
       \hbox to 6.28in { {\it Professores: {\rm #3} \hfill Name: {\rm #5} } }
       %\hbox to 6.28in { {\it TA: #4  \hfill #6}}
      \vspace{2mm}}
   }
   \end{center}
   \markboth{#5 -- #1}{#5 -- #1}
   \vspace*{4mm}
}

\newcommand{\problem}[2]{~\\\fbox{\textbf{#1}}\hfill \newline\newline}
\newcommand{\subproblem}[1]{~\newline\textbf{(#1)}}
\newcommand{\D}{\mathcal{D}}
\newcommand{\Hy}{\mathcal{H}}
\newcommand{\VS}{\textrm{VS}}
\newcommand{\solution}{~\newline\textbf{\textit{(Solução)}} }

\newcommand{\bbF}{\mathbb{F}}
\newcommand{\bbX}{\mathbb{X}}
\newcommand{\bI}{\mathbf{I}}
\newcommand{\bX}{\mathbf{X}}
\newcommand{\bY}{\mathbf{Y}}
\newcommand{\bepsilon}{\boldsymbol{\epsilon}}
\newcommand{\balpha}{\boldsymbol{\alpha}}
\newcommand{\bbeta}{\boldsymbol{\beta}}
\newcommand{\0}{\mathbf{0}}
%%%% macros.tex

\begin{document}
\homework{Exercício - Cinemática Relativística em Física de Partículas}{28/10/2024}{Dilson, Eliza \& Maurício}{}{Diogo Gomes Santos Caffonso de Morais}

\problem{EXERCICIO}

O código utilizado está abaixo. Foi executado de dentro do JupyterNotebook.

\begin{lstlisting}
#include <TFile.h>
#include <TTree.h>
#include <TTreeReader.h>
#include <TTreeReaderArray.h>
#include <TCanvas.h>
#include <TH1F.h>
#include <TMath.h>
#include <iostream>
#include <vector>
#include <filesystem>

double calcular_massa_invariante(const TTreeReaderArray<float>& pt, const TTreeReaderArray<float>& eta, const TTreeReaderArray<float>& phi) {
    if (pt.GetSize() >= 2) {
        return sqrt(2 * pt[0] * pt[1] * (TMath::CosH(eta[0] - eta[1]) - TMath::Cos(phi[0] - phi[1])));
    }
    return -1.0;  // Valor inválido caso não haja pelo menos dois elementos
}

void analise() {
    std::vector<std::string> diretorios = {
        "/opendata/eos/opendata/cms/Run2016G/DoubleEG/NANOAOD/UL2016_MiniAODv2_NanoAODv9-v1/100000",
        "/opendata/eos/opendata/cms/Run2016G/DoubleEG/NANOAOD/UL2016_MiniAODv2_NanoAODv9-v1/1010000",
        "/opendata/eos/opendata/cms/Run2016G/DoubleEG/NANOAOD/UL2016_MiniAODv2_NanoAODv9-v1/250000"
    };

    std::vector<double> e_massas_invariantes, m_massas_invariantes, t_massas_invariantes;

    for (const auto& dir : diretorios) {
        for (const auto& entry : std::filesystem::directory_iterator(dir)) {
            std::string file_path = entry.path();
            TFile file(file_path.c_str(), "READ");
            if (!file.IsOpen()) continue;

            TTreeReader reader("Events", &file);
            TTreeReaderArray<float> Electron_pt(reader, "Electron_pt");
            TTreeReaderArray<float> Electron_eta(reader, "Electron_eta");
            TTreeReaderArray<float> Electron_phi(reader, "Electron_phi");
            TTreeReaderArray<float> Muon_pt(reader, "Muon_pt");
            TTreeReaderArray<float> Muon_eta(reader, "Muon_eta");
            TTreeReaderArray<float> Muon_phi(reader, "Muon_phi");
            TTreeReaderArray<float> Tau_pt(reader, "Tau_pt");
            TTreeReaderArray<float> Tau_eta(reader, "Tau_eta");
            TTreeReaderArray<float> Tau_phi(reader, "Tau_phi");

            while (reader.Next()) {
                if (Electron_pt.GetSize() >= 2) {
                    e_massas_invariantes.push_back(calcular_massa_invariante(Electron_pt, Electron_eta, Electron_phi));
                }
                if (Muon_pt.GetSize() >= 2) {
                    m_massas_invariantes.push_back(calcular_massa_invariante(Muon_pt, Muon_eta, Muon_phi));
                }
                if (Tau_pt.GetSize() >= 2) {
                    t_massas_invariantes.push_back(calcular_massa_invariante(Tau_pt, Tau_eta, Tau_phi));
                }
            }
        }
    }

    TCanvas* canvas = new TCanvas("canvas", "Distribuições de Massas Invariantes", 800, 600);
    TH1F* hEletron = new TH1F("hEletron", "", 50, 0, 200);
    TH1F* hMuon = new TH1F("hMuon", "", 50, 0, 200);
    TH1F* hTau = new TH1F("hTau", "", 50, 0, 200);

    for (const auto& massa : e_massas_invariantes) if (massa >= 0) hEletron->Fill(massa);
    for (const auto& massa : m_massas_invariantes) if (massa >= 0) hMuon->Fill(massa);
    for (const auto& massa : t_massas_invariantes) if (massa >= 0) hTau->Fill(massa);

    hEletron->SetLineColor(kBlue);
    hEletron->SetStats(0); // Desativa a caixa de estatísticas
    hEletron->GetXaxis()->SetTitle("e_mass (GeV/c^{2})"); // Rótulo do eixo X
    hEletron->GetYaxis()->SetTitle("Eventos"); // Rótulo do eixo Y
    hEletron->Draw();
    canvas->SaveAs("e_massa_invariante.png");

    hMuon->SetLineColor(kBlue);
    hMuon->SetStats(0); // Desativa a caixa de estatísticas
    hMuon->GetXaxis()->SetTitle("#mu_mass (GeV/c^{2})"); // Rótulo do eixo X
    hMuon->GetYaxis()->SetTitle("Eventos"); // Rótulo do eixo Y
    hMuon->Draw();
    canvas->SaveAs("m_massa_invariante.png");

    hTau->SetLineColor(kBlue);
    hTau->SetStats(0); // Desativa a caixa de estatísticas
    hTau->GetXaxis()->SetTitle("#tau_mass (GeV/c^{2})"); // Rótulo do eixo X
    hTau->GetYaxis()->SetTitle("Eventos"); // Rótulo do eixo Y
    hTau->Draw();
    canvas->SaveAs("t_massa_invariante.png");

    delete hEletron;
    delete hMuon;
    delete hTau;
    delete canvas;
}

\end{lstlisting}

Executado o código acima, os histogramas obtidos foram os publicados \href{https://github.com/DiogoCaffonso/FAE_Repo/tree/master/Trabalho5/exercicio/Images/SegundoPasso}{neste repositório GitHub}.

\end{document} 