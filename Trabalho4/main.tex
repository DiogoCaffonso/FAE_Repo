\documentclass[a4 paper]{article}
\usepackage[brazil]{babel}
\usepackage[utf8]{inputenc}
% Set target color model to RGB
\usepackage[inner=2.0cm,outer=2.0cm,top=2.5cm,bottom=2.5cm]{geometry}
\usepackage{setspace}
\usepackage[rgb]{xcolor}
\usepackage{verbatim}
\usepackage{subcaption}
\usepackage{amsgen,amsmath,amstext,amsbsy,amsopn,amssymb}
\usepackage{fancyhdr}
\usepackage[colorlinks=true, urlcolor=blue,  linkcolor=blue, citecolor=blue]{hyperref}
\usepackage[colorinlistoftodos]{todonotes}
\usepackage{rotating}
%\usetikzlibrary{through,backgrounds}
\hypersetup{%
pdfauthor={Ashudeep Singh},%
pdftitle={Homework},%
pdfkeywords={Tikz,latex,bootstrap,uncertaintes},%
pdfcreator={PDFLaTeX},%
pdfproducer={PDFLaTeX},%
}
%\usetikzlibrary{shadows}
% \usepackage[francais]{babel}
\usepackage{booktabs}
\usepackage{xcolor}
% Definindo novas cores
\definecolor{verde}{rgb}{0,0.5,0}
% Configurando layout para mostrar codigos C++
\usepackage{listings}
\lstset{
  language=C++,
  basicstyle=\ttfamily\small, 
  keywordstyle=\color{blue}, 
  stringstyle=\color{verde}, 
  commentstyle=\color{red}, 
  extendedchars=true, 
  showspaces=false, 
  showstringspaces=false, 
  numbers=left,
  numberstyle=\tiny,
  breaklines=true, 
  backgroundcolor=\color{green!10},
  breakautoindent=true, 
  captionpos=b,
  xleftmargin=0pt,
}

%%%%%% macros.tex
\newtheorem{thm}{Theorem}[section]
\newtheorem{prop}[thm]{Proposition}
\newtheorem{lem}[thm]{Lemma}
\newtheorem{cor}[thm]{Corollary}
\newtheorem{defn}[thm]{Definition}
\newtheorem{rem}[thm]{Remark}
\numberwithin{equation}{section}

\newcommand{\homework}[6]{
   \pagestyle{myheadings}
   \thispagestyle{plain}
   \newpage
   \setcounter{page}{1}
   \noindent
   \begin{center}
   \framebox{
      \vbox{\vspace{2mm}
    \hbox to 6.28in { {\bf Introdução à Análise de dados em FAE \hfill {\small (#2)}} }
       \vspace{6mm}
       \hbox to 6.28in { {\Large \hfill #1  \hfill} }
       \vspace{6mm}
       \hbox to 6.28in { {\it Professores: {\rm #3} \hfill Name: {\rm #5} } }
       %\hbox to 6.28in { {\it TA: #4  \hfill #6}}
      \vspace{2mm}}
   }
   \end{center}
   \markboth{#5 -- #1}{#5 -- #1}
   \vspace*{4mm}
}

\newcommand{\problem}[2]{~\\\fbox{\textbf{#1}}\hfill \newline\newline}
\newcommand{\subproblem}[1]{~\newline\textbf{(#1)}}
\newcommand{\D}{\mathcal{D}}
\newcommand{\Hy}{\mathcal{H}}
\newcommand{\VS}{\textrm{VS}}
\newcommand{\solution}{~\newline\textbf{\textit{(Solução)}} }

\newcommand{\bbF}{\mathbb{F}}
\newcommand{\bbX}{\mathbb{X}}
\newcommand{\bI}{\mathbf{I}}
\newcommand{\bX}{\mathbf{X}}
\newcommand{\bY}{\mathbf{Y}}
\newcommand{\bepsilon}{\boldsymbol{\epsilon}}
\newcommand{\balpha}{\boldsymbol{\alpha}}
\newcommand{\bbeta}{\boldsymbol{\beta}}
\newcommand{\0}{\mathbf{0}}
%%%% macros.tex

\begin{document}
\homework{Exercícios - Aula ROOT}{15/10/2024}{Dilson, Eliza \& Maurício}{}{Diogo Gomes Santos Caffonso de Morais}

\problem{EXERCICIO 1 }

Para o primeiro exercício, foi desenvolvido o código que segue:

\begin{lstlisting}
#include <RooRealVar.h>
#include <RooCBShape.h>
#include <RooDataSet.h>
#include <RooPlot.h>
#include <TCanvas.h>
#include <TStyle.h>
#include <TPaveText.h>

void exercicio1() {
	RooRealVar x("x", "x", -10, 10);
	RooRealVar mean("mean", "Mean", 0, -10, 10);
	RooRealVar sigma("sigma", "Sigma", 1, 0.1, 5);
	RooRealVar alpha("alpha", "Alpha", 1.5, 0.1, 5);
	RooRealVar n("n", "n", 2, 0.1, 10);

	RooCBShape crystalBall("crystalBall", "Crystal Ball PDF", x, mean, sigma, alpha, n);

	RooDataSet* data = crystalBall.generate(x, 10000);

	crystalBall.fitTo(*data);

	RooPlot* xframe = x.frame();
	data->plotOn(xframe);
	crystalBall.plotOn(xframe);

	TCanvas* c1 = new TCanvas("c1", "Ajuste Crystal Ball", 800, 600);
	xframe->Draw();

	gStyle->SetOptStat(1111);
	gStyle->SetOptFit(1111);

	TPaveText* paramsText = new TPaveText(0.6, 0.7, 0.9, 0.9, "NDC");
	paramsText->AddText(Form("Mean ajustado = %.3f", mean.getVal()));
	paramsText->AddText(Form("Sigma ajustado = %.3f", sigma.getVal()));
	paramsText->AddText(Form("Alpha ajustado = %.3f", alpha.getVal()));
	paramsText->AddText(Form("n ajustado = %.3f", n.getVal()));
	paramsText->SetFillColor(0);
	paramsText->SetTextSize(0.03);
	paramsText->Draw();

	c1->SaveAs("exercicio1.png");
}
\end{lstlisting}

Executado o código acima, tive o seguinte output:

\includegraphics[width=\linewidth]{ex1.png}
\begin{center}
    Figura 1: Gráfico da função de fit.
\end{center}

\newpage

\problem{EXERCICIO 2 }

Para o segundo exercício, foi desenvolvido o código que segue:

\begin{lstlisting}
#include <iostream>
#include <cmath>
#include "TH1F.h"
#include "TF1.h"
#include "TCanvas.h"
#include "TRandom.h"

void exercicio2() {

	int nEvents = 1500;
	double lambdaTrue = 1.0;
	double x[nEvents];

	for (int i = 0; i < nEvents; i++) {
		x[i] = -log(1 - gRandom->Uniform()) / lambdaTrue;
	}

	TH1F *hist = new TH1F("hist", "Histograma de eventos simulados", 50, 0, 10);
	for (int i = 0; i < nEvents; i++) {
		hist->Fill(x[i]);
	}

	TF1 *fitFunction = new TF1("fitFunction", "[0] * exp(-[1] * x)", 0, 10);
	fitFunction->SetParameters(nEvents, 1.0);
	fitFunction->SetParLimits(1, 0.1, 2);
	hist->Fit("fitFunction", "R");

	double lambdaAdjusted = fitFunction->GetParameter(1);
	double totalYield = fitFunction->Integral(0, 10);
 
	TCanvas *c1 = new TCanvas("c1", "Ajuste da função exponencial", 800, 600);
	hist->Draw();
	fitFunction->Draw("same");
	std::cout << "Valor ajustado para lambda: " << lambdaAdjusted << std::endl;

	c1->SaveAs("exercicio2.png");
}
\end{lstlisting}
Que gerou o gráfico:

\includegraphics[width=\linewidth]{ex2.png}
\begin{center}
    Figura 2: Histograma com função de fit.
\end{center}

Respondendo às perguntas:

$\lambda=1,03$

$Número total de eventos ajustados: 1500$

Estão dentro das expectativas, já que não desviam muito dos valores propostos.

\newpage

\problem{EXERCICIO 3 }

Para o terceiro exercício, foi desenvolvido o código que segue:

\begin{lstlisting}
#include <iostream>
#include "RooDataSet.h"
#include "RooRealVar.h"
#include "RooCBShape.h"
#include "RooPolynomial.h"
#include "RooAddPdf.h"
#include "RooPlot.h"
#include "TFile.h"
#include "TCanvas.h"
#include "TLegend.h"

void exercicio3() {
	TFile *file = TFile::Open("/home/cms-opendata/FAE_Repo/Trabalho4/ex3/DataSet_lowstat.root");
	RooDataSet *data = dynamic_cast<RooDataSet*>(file->Get("data"));

	RooRealVar mass("mass", "Massa", 2.8, 3.5);
	RooRealVar mean("mean", "Média", 3.096916, 3.0, 3.2);
	RooRealVar sigma("sigma", "Desvio Padrão", 0.05, 0.01, 0.1);
	RooRealVar alpha("alpha", "Alpha", 1.5);
	RooRealVar n("n", "n", 0.5, 0, 5);
	RooCBShape signal("signal", "Sinal J/ψ", mass, mean, sigma, alpha, n);
    
	RooRealVar a0("a0", "a0", 0, -10, 10);
	RooRealVar a1("a1", "a1", 1000, 0, 10000);
	RooPolynomial background("background", "Fundo", mass, RooArgList(a0, a1));
    
	RooRealVar nsig("nsig", "Número de eventos sinal", 1000, 0, 10000);
	RooRealVar nbkg("nbkg", "Número de eventos fundo", 1000, 0, 10000);
	RooAddPdf model("model", "Modelo Sinal + Fundo", RooArgList(signal, background), RooArgList(nsig, nbkg));
    
	model.fitTo(*data);
	RooPlot *frame = mass.frame();
	frame->SetXTitle("Massa (GeV/c^{2})");

	data->plotOn(frame);
	model.plotOn(frame);
	model.paramOn(frame);

	double chi2 = frame->chiSquare();
	int nParams = model.getParameters(*data)->getSize();
	int nPoints = data->numEntries();
	int ndf = nPoints - nParams;

	std::cout << "Chi2/ndf: " << chi2 / ndf << std::endl;

	TCanvas *c1 = new TCanvas("c1", "Ajuste da ressonância J/ψ", 800, 600);
	frame->Draw();

	c1->SaveAs("exercicio3.png");
}
\end{lstlisting}
Que gerou o gráfico:

\includegraphics[width=\linewidth]{ex3.png}
\begin{center}
    Figura 3: Gráfico de dispersão função de fit.
\end{center}

\large $$\chi^{2}/ndf=2,68753\times 10^{-3}$$ \normalsize

Obs.: A razão $\chi^{2}/ndf$ está muito pequena, o que pode levar a um "excesso" de ajuste.

\end{document} 